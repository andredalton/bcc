\documentclass[12pt,a4paper]{article}
\usepackage[brazil]{babel}
\usepackage[utf8]{inputenc}
\usepackage[tmargin=2cm, bmargin=2cm, lmargin=2cm, rmargin=2cm]{geometry}
\usepackage{graphicx}
\usepackage{float}
\usepackage{indentfirst}
\usepackage[pdftex]{hyperref}
\usepackage{enumerate}
\usepackage{bbm,amsthm,amssymb,amstext,amsmath}

\author{William Gnann - 6431176}
\title{Relatório: Fatoração QR}

\begin{document}
\maketitle
\section{Compilação, Sistema e Implementação}
O exercício programa entregue, bem como quaisquer testes, foram realizados num 
sistema Debian GNU/Linux compatível com a GNU/Linux Pró-Aluno.

\subsection{Compilação}
Junto com o código, há um \verb+Makefile+. A compilação é feita usando o 
comando \verb+make+. Caso não houver o make, basta utilizar \verb+gcc main.c qr.c -Wall -ansi -lm -o qr+.

\subsection{Como utilizar}
Para rodar o programa, após tê-lo compilado, executar:
\begin{quote}
\begin{verbatim}
qr [MODO] < [entrada]
\end{verbatim}
\end{quote}

Onde \verb+[MODO]+ é o número referente ao exemplo e \verb+[entrada]+ é o arquivo que com os dados.

Os arquivos estão localizados no diretório \verb+exemplos+.

OBS: os dados são obtidos a partir da entrada padrão. 

\subsection{A fatoração QR}
Resolvi utilizar a macro \verb+DBL_EPSILON+ que está em \verb+float.h+ para definir a condição de parada. 
Isso faz mais sentido do que verificar se é igual à zero.

Além disso, utilizei a reorganização de contas para melhorar a estabilidade numérica \cite{GSM}.

\section{Exemplo 1}
Nesta seção, algumas considerações e respostas sobre o primeiro exemplo.
\subsection{Distância}
A distância entre $Ax$ e $b$ é justamente a norma do vetor $b-Ax$ que calculamos com o programa.
Seu valor é aproximadamente $1.82216$.

\pagebreak
\section{Exemplo 2}
Nesta seção, algumas considerações e respostas sobre o segundo exemplo.
\subsection{Formulação do problema}
Dado o polinômio cúbico $c_0 + c_1s + c_2s^2 + c_3s^3$, obtemos o sistema:

\[
\left[
\begin{array}{cccc}
    1 & s_1 & s_{1}^{2} & s_{1}^{3} \\
      & ... \\
    1 & s_{11} & s_{11}^{2} & s_{11}^{3} \\
\end{array}
\right]
\left[
\begin{array}{c}
    c_0 \\
    c_1 \\
    c_2 \\
    c_3 \\
\end{array}
\right]
=
\left[
\begin{array}{c}
    75.995 \\
    ... \\
    281.422 \\
\end{array}
\right]
\]

Onde $s_i$ é $(t_i-1950)/50$. Substituindo, temos aproximadamente:

\[
\left[
\begin{array}{cccc}
    1.00 & -1.00 & 1.00 & -1.00 \\
    1.00 & -0.80 & 0.64 & -0.51 \\
    1.00 & -0.60 & 0.36 & -0.22 \\
    1.00 & -0.40 & 0.16 & -0.06 \\
    1.00 & -0.20 & 0.04 & -0.01 \\
    1.00 & 0.00 & 0.00 & 0.00 \\
    1.00 & 0.20 & 0.04 & 0.01 \\
    1.00 & 0.40 & 0.16 & 0.06 \\
    1.00 & 0.60 & 0.36 & 0.22 \\
    1.00 & 0.80 & 0.64 & 0.51 \\
    1.00 & 1.00 & 1.00 & 1.00 \\
\end{array}
\right]
\left[
\begin{array}{c}
    c_0 \\
    c_1 \\
    c_2 \\
    c_3 \\
\end{array}
\right]
=
\left[
\begin{array}{c}
    75.995 \\
    91.972 \\
    105.711 \\
    123.203 \\
    131.669 \\
    150.697 \\
    279.323 \\
    203.212 \\
    226.505 \\
    249.633 \\
    281.422 \\
\end{array}
\right]
\]

\subsection{Resultado}
Resolvendo o sistema, obtemos o vetor aproximado:
\[
\left[
\begin{array}{c}
    155.90427 \\
    100.36592 \\
    23.72614 \\
    1.26294 \\
\end{array}
\right]
\]

Donde, fazendo as devidas substituições, obtemos o valor aproximado $312.691$ 
para o ano de $2010$. O gráfico abaixo dá uma ideia melhor sobre o ajuste da 
função aos pontos.

\begin{figure}[H]\centering
    \includegraphics[scale=0.5]{polinomio.jpg}
    \caption{Ajuste dos pontos ao polinômio.}
\end{figure}

\section{Exemplo 3}
Nesta seção, algumas considerações e respostas sobre o terceiro exemplo.
\subsection{Formulação do problema}
Dado $z = 0$ e $f = 1$, obtemos o polinômio de duas variáveis $ax^2 + bxy + cy^2 + dx + ey$ que resulta no sistema:

\[
\left[
\begin{array}{ccccc}
    x_{1}^2 & x_{1}y_{1} & y_{1}^{2} & x_{1} & y_{1} \\
     & & ... \\
    x_{10}^2 & x_{10}y_{10} & y_{10}^{2} & x_{10} & y_{10} \\
\end{array}
\right]
\left[
\begin{array}{c}
    a \\
    b \\
    c \\
    d \\
    e \\
\end{array}
\right]
=
\left[
\begin{array}{c}
    -1 \\
    ... \\
    -1 \\
\end{array}
\right]
\]

Substituindo, temos aproximadamente:

\[
\left[
\begin{array}{ccccc}
    1.04 & 0.40 & 0.15 & 1.02 & 0.39 \\
    0.90 & 0.30 & 0.10 & 0.95 & 0.32 \\
    0.76 & 0.23 & 0.07 & 0.87 & 0.27 \\
    0.59 & 0.17 & 0.05 & 0.77 & 0.22 \\
    0.45 & 0.12 & 0.03 & 0.67 & 0.18 \\
    0.31 & 0.08 & 0.02 & 0.56 & 0.15 \\
    0.19 & 0.06 & 0.02 & 0.44 & 0.13 \\
    0.09 & 0.04 & 0.01 & 0.30 & 0.12 \\
    0.03 & 0.02 & 0.02 & 0.16 & 0.13 \\
    0.00 & 0.00 & 0.02 & 0.01 & 0.15 \\
\end{array}
\right]
\left[
\begin{array}{c}
    a \\
    b \\
    c \\
    d \\
    e \\
\end{array}
\right]
=
\left[
\begin{array}{c}
    -1 \\
    -1 \\
    -1 \\
    -1 \\
    -1 \\
    -1 \\
    -1 \\
    -1 \\
    -1 \\
    -1 \\
\end{array}
\right]
\]

\subsection{Resultado}
Resolvendo o sistema, obtemos o vetor aproximado:
\[
\left[
\begin{array}{c}
    2.25379 \\
    0.00632 \\
    5.52218 \\
    -1.28981 \\
    -7.37735 \\
\end{array}
\right]
\]

O erro obtido foi de aproximadamente $0.02778$. Isso significa que o ajuste 
deverá ser razoavelmente bom. A imagem evidencia a qualidade do ajuste.

\begin{figure}[ht]\centering
    \includegraphics[scale=0.5]{elipse.jpg}
    \caption{Ajuste da elipse gerada a partir do MMQ}
\end{figure}

\begin{thebibliography}{9999}
    \bibitem[GSM]{GSM}Wikipedia, \textsl{Gram-Schmidt process},\\
        \mbox{}\hfill\texttt{http://en.wikipedia.org/wiki/Gram-Schmidt}
\end{thebibliography}
\end{document}
