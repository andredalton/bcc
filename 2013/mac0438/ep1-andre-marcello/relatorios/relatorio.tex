\documentclass[12pt,a4paper]{article}
\usepackage[brazil]{babel}
\usepackage[utf8]{inputenc}
\usepackage[tmargin=2cm, bmargin=2cm, lmargin=2cm, rmargin=2cm]{geometry}
\usepackage{graphicx}
\usepackage{float}
\usepackage{indentfirst}
\usepackage[pdftex]{hyperref}
\usepackage{enumerate}
\usepackage{bbm,amsthm,amssymb,amstext,amsmath}
\usepackage{verbatim}

\author{
    André Meneghelli Vale - 4898948\\
    \texttt{andredalton@gmail.com}
    \and
    Marcello Souza de Oliveira - 6432692\\
    \texttt{mcellor210@gmail.com}
}

\title{Relatório: IronMan}

\begin{document}
\maketitle

\section{Descrição do problema, Abordagem, Algorítmo da punição}

\subsection{Descrição do problema}

   O presente Exercício Programa tem por finalidade simular o \emph{Campeonato Mundial 
de \'Iron Man\'}. Neste, o triatleta deve ser capaz de percorrer 3.8 Km de natação,
seguido de 180 Km de ciclismo e por fim 42 Km de corrida. Sendo que a estrada do ciclismo possui
trechos planos, descidas e de subidas. Esses competidores são usualmente
divididos entre homens profissionais (hp), mulheres profissionais (mp), homens
amadores (ha) e mulheres amadoras (ma). \\
Este simulador, parte da solução concorrente em \verb+linguagem \'C\'+ e faz uso da
biblioteca \verb+\'pthread.h\'+, para alocação e gerênciamento de \verb+POSIX threads+ para
Linux. A idéia básica é criar hp + mp + ha + ma threads atletas e uma thread que
gerencie a classificação destes e limitar o paralelismo desses processos
pelas seguintes condições:

\begin{itemize}
\item Entre a fase de natação e ciclismo, existe um trajeto de transição com um
portal na entrada e um portal na saída, nos quais não se pode passar em paralelo
(fila indiana) e denominado transição T1;
\item Entre a fase de ciclismo e corrida, existe outro trajeto de transição
similar ao primeiro e denominado T2;
\item Na fase de ciclismo, não pode haver mais que 3 atletas lado a lado (mesmo
tempo e posição - trajeto de 3 faixas);
\item Cada fase possui um intervalo de tempo ou velocidade para conclusão por
unidade de 100 m na natação e 1 Km nos demais.
\item A cada 30 min ou 1 min (debug) de corrida o programa deve fornecer
dados referentes a posição de cada participante e por fim a classificação
final com os tempos destes atletas.
\end{itemize}

\subsection{Abordagem}

A intenção do programa é a paralelização do código, que não é totalmente 
possível devido as seções críticas geradas pelas restrições e modelagem do 
problema (forma como o problema é tratado: pode levar a mais seções críticas 
que o necessário, muito embora não seja um problema trivial de se analizar). Isso
nos leva a um problema de concorrência em tais seções. Para enfatizar tais 
diferênças, criamos executáveis sem as restrições ("totalmente" paralelo), sem 
paralelismo (versão iterativa) e com o problema da concorrência. \\
Claramente o problema principal de concorrência se dá na fase do ciclismo,
onde tem-se a restrição de ultrapassagem, e a punição devido ao número máximo de
faixas possível para os atletas usarem). \\
Durante o processo de modelagem do problema das seções críticas, chegamos a 
uma solução que nos pareceu (mediante testes) satisfatoriamente eficiente.
A solução adotada foi a alocação de grande quantidade de memória RAM a fim de
criar \verb+buffers+ de acesso ordenados. Deste modo a maioria dos cálculos encolvendo
os atletas se dá em variáveis alocadas previamente para cada uma das \verb+threads+
específicas. \\
A outra medida adotada foi a criação de um algoritmo de busca e inserção
chamado de \verb+punição+ que será tratado a seguir. Com isso, o problema inicialmente
complexo da punição, torna-se uma recursão de buscas binárias e inserções em vetor
ordenado (gerenciado por um semáforo que permite a escrita).

\subsection{Algorítimo da punição}

A punicaoR(), busca recursivamente um tempo igual ao parâmetro de entrada, em um vetor
com a posição de atletas (inicializado com -1), como o vetor está ordenado por
construção, a busca é rápida. Porém a cada punição que precisa ser realizada, uma nova
busca deve ser feita. No entanto esta busca pode ser feita em um subconjunto bem menor
que o original. \\
Como os valores que interessam ao programa após uma punição estão todos à direita
e não podem ter tamanho superior ao número de faixas vezes o tamanho da punição então
basta restringir o subconjunto a um conjunto de tamanho \verb+NFAIXAS * PUNICAO+ deslocado
para direita até a primeira posição cujo valor é maior que o tempo atual do atleta. \\ 
Estas restrições na busca, bem como ela já estar ordenada de início fazem com que a complexidade
do algoritmo seja, em média, pouco maior que logN.\\

\pagebreak
\section{Compilação, Sistema e Implementação}
O exercício programa entregue, bem como quaisquer testes, foram realizados em sistemas baseados em Debian com diversos tipos de arquiteruras diferentes e em um servidor instalado com SUSE Linux Enterprise Server 11.

\subsection{Compilação}
Junto com o código, há um \verb+Makefile+. A compilação é feita usando o 
comando \verb+make+. Por praticidade foram criandas duas targets extras \verb+all+ e \verb+clear+ que compilam todos os executáveis e limpa todo o conteúdo gerado na compilação automaticamente. Também é possível gerar um relatório automático que atualiza este relatório com os valores calculados para o computador do leitor. Para tanto basta executar o \verb+./relatorio.sh+.

\subsection{Como utilizar}
Para rodar o programa, após tê-lo compilado, executar:
\begin{quote}
\begin{verbatim}
./ep1 -debug [entrada]
\end{verbatim}
\end{quote}

Onde:

\verb+-debug+ significa que o programa irá rodar em modo de depuração e imprimirá uma linha com a posição de todos os atletas a cada minuto. Caso contrário irá imprimir os três primeiros colocados de cada categoria a cada 30 minutos.

\verb+[entrada]+ um arquivo que contém os dados sobre a quantidade de atletas e a condição do terreno na prova de ciclismo.

OBS: Os arquivos estão localizados no diretório \verb+entradas/+. O formato do arquivo de entrada deve seguir a descrição do enunciado do problema em \verb+ep1.pdf+, não foram tratados erros de formatação do arquivo de entrada.

\subsection{Códigos}
O código foi fragmentado em 3 arquivos .c principais e mais dois arquivos de cabeçalho:

\begin{itemize}
\item \verb+ep1.c+: Contém as duas funções que definem a classificação e os atletas e mais duas funções principais, as quais são escolhidas em tempo de compilação:
\begin{itemize}
\item \verb+ironMain+: É a função principal para a versão \textit{paralela} e para a versão \textit{concorrente} do problema.
\item \verb+gutsMain+: É a função principal para a versão \textit{iterativa} do problema.
\end{itemize}
\item \verb+tempos.c+ e \verb+tempos.h+: Contém a geração aleatória dos tempos de cada atleta, respeitando os limites estipulados no enunciado do problema.
\item \verb+defines.c+ e \verb+defines.h+:  Contém a maioria dos defines, as estruturas de dados usadas no problema e ainda uma função principal para realizar uma simulação interativa do algorítimo da punição. A utilização deste executável será tratada adiante.
\end{itemize}

\subsection{Executáveis}

Foram desenvolvidos 3 algorítmos diferentes para solucionar o problema e cada um destes algorítimos foi implementado com precisão de segundos e milisegundos além de mais dois executáveis extras para auxilio do entendimento do problema. A seguir segue uma breve descrilção de cada um deles: 

\begin{itemize}
\item \verb+ep1+: Programa principal. Roda a solução \textit{concorrente} com precisão de 1s.
\item \verb+ep1-mili+: Mesmo algoritmo do \verb+ep1+ porém rodando com precisão de 1ms. Essa precisão diminui consideravelmente o número de colisões.
\item \verb+pep1+: Roda a solução totalmente em paralelo com precisão de 1s. Portanto não é capaz de calcular as punições. Foi a primeira solução desenvolvida e foi mantida para ser usada como ferramenta de análise dos efeitos da concorrência.
\item \verb+pep1-mili+: Mesmo algoritmo de \verb+pep1+ porém com precisão de 1ms.
\item \verb+gep1+: Após o desenvolvimento da função de punição tratar o problema de modo iterativo se tornou possível. Este executável tem o objetivo de testar os efeitos da execução do programa em apenas um núcleo de processamento. Tem precisão de 1s.
\item \verb+gep1-mili+: Exatamente igual a \verb+gep1+, porém com precisão de 1ms.
\item \verb+ep1-sleep+: Exatamente igual a \verb+ep1+, no entanto pausa a saída do programa em 1 décimo de segundo após a ultima impressão de resultados.
\item \verb+simula+: Executável que simula o \textbf{algoritimo da punição}. Este programa pode receber \verb+-h+ para imprimir uma ajuda ou então dois números naturais que representam a quantidade de faixas disponível e a quantidade de atletas que esta participando do evento.
\end{itemize}

\section{Testes}
Aqui serão expostos os arquivos de teste utilizados e os resultados finais.
\subsection{Arquivos de entrada}

\begin{itemize}
\item \verb+arquivo1.txt+: \href{run:./re1.txt}{Saida 1} \\*
\texttt{5 5 20 20 P 50 S 10 D 20 P 100}
\item \verb+arquivo2.txt+: \href{run:./re2.txt}{Saida 2} \\*
\texttt{20 15 100 65 P 10 S 10 D 10 P 20 S 6 D 10 P 20 S 10 D 10 P 30 S 10 D 10 P 24}
\item \verb+arquivo3.txt+: \href{run:./re3.txt}{Saida 3} \\*
\texttt{500 500 500 500 P 50 S 10 D 20 P 100}
\item \verb+arquivo4.txt+: \href{run:./re4.txt}{Saida 4} \\*
\texttt{5000 5000 5000 5000 P 10 S 10 D 10 P 20 S 6 D 10 P 20 S 10 D 10 P 30 S 10 D 10 P 24}
\item \verb+arquivo5.txt+: Este arquivo teve muitos problemas e não foi executado na maioria dos testes. \\*
\texttt{25000 25000 25000 25000 P 10 S 10 D 10 P 20 S 6 D 10 P 20 S 10 D 10 P 30 S 10 D 10 P 24}
\end{itemize}

\subsection{Arquiteturas}

\begin{table}
\begin{tabular}{|l|l|l|l|l|l|}
  \hline
  \textbf{Nome} & \textbf{Processador} & \textbf{Núcleos} & \textbf{Clock} & \textbf{RAM} & \textbf{Arquitetura} \\
  \hline
  \input{maquinas.tex}
  \input{mycpu.tex}
\end{tabular}
\end{table}

\pagebreak

\begin{table}
\begin{tabular}{|l|l|l|l|l|l|}
  \hline
  \textbf{Nome} & \textbf{Programa} & \textbf{Arquivo} & \textbf{Tempo} & \textbf{RAM} & \textbf{Saída} \\
  \hline
  \input{tempos.tex}
  \input{myep1.tex}
\end{tabular}
\end{table}

\section{Conclusão}
Um programa paralelizado, quando da posse de uma máquina multi-thread,
é muito mais eficiente que um interativo. Permitindo assim que todo o potencial
destas máquinas seja usado. Provavelmente o futuro da programação está no processamento
paralelo. No entanto uma pequena parte de problemas tem um caráter paralelizável,
foi o que pode ser constatado neste exercício. \\
Nestes casos o objetivo do programa passa a ser reduzir ao máximo o número de seções 
críticas, permitindo assim que os processos não fiquem ociosos. Porém essa tarefa não é
fácil e temos que tomar os devidos cuidados para que nenhuma restrição seja mal projetada.\\
A idéia de minimizar a sessão crítica para pontos teve um alto custo em memória RAM. Deixando
claro que soluções paralelas são soluções de grande potencial mas que para gerar bons resultados
envolvem um alto investimento em hardware.

\end{document}
