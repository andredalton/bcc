\documentclass[12pt,a4paper]{article}
\usepackage[brazil]{babel}
\usepackage[utf8]{inputenc}
\usepackage[tmargin=2cm, bmargin=2cm, lmargin=2cm, rmargin=2cm]{geometry}
\usepackage{graphicx}
\usepackage{float}
\usepackage{indentfirst}
\usepackage[pdftex]{hyperref}
\usepackage{enumerate}
\usepackage{amsthm,amssymb,amstext,amsmath}
\usepackage{verbatim}
\usepackage{amsmath}
\usepackage{eucal}
\usepackage{mathrsfs}
\usepackage[utf8]{inputenc}
\usepackage{hyperref}
\usepackage{fancyvrb}
\DefineVerbatimEnvironment{code}{Verbatim}{fontsize=\small}
\DefineVerbatimEnvironment{example}{Verbatim}{fontsize=\small}

\title{Título\\MAC0438 - Programação Concorrente}
\author{
    André Meneghelli Vale - 4898948\\
    \texttt{andredalton@gmail.com}
    \and
    Marcello Souza de Oliveira - 6432692\\
    \texttt{mcellor210@gmail.com}
}
\date{}

\pdfinfo{%
  /Title    ()
  /Author   ()
  /Creator  ()
  /Producer ()
  /Subject  ()
  /Keywords ()
}

\begin{document}
\maketitle

\newpage

\section{Cálculo por série infinita de $\pi$}

\subsection{Fórmula:}
A fórmula escolhida foi a \emph{Fórmula de Bellard}\textsuperscript{[1]}. Esta escolha se deve ao fato de poder ser calculado um enésimo dígito de pi na base 2 sem a necessidade de se calcular qualquer um dos termos anteriores, o que torna o programa possível de ser totalmente paralelizável.
Esta fórmula foi desenvolvida a partir da fórmula \emph{Bailey–Borwein–Plouffe}\textsuperscript{[2]} (\verb+BBP+) que por sua vez foi inspirada na série infinita para o cálculo de uma aproximação da inversa da arcotangente.

\begin{align}
\pi = \frac1{2^6} \sum_{n=0}^\infty \frac{(-1)^n}{2^{10n}} \, \left(-\frac{2^5}{4n+1} \right. & {} - \frac1{4n+3} + \frac{2^8}{10n+1} - \frac{2^6}{10n+3} \left. {} - \frac{2^2}{10n+5} - \frac{2^2}{10n+7} + \frac1{10n+9} \right)
\end{align}

\begin{align}
\pi = \sum_{i = 0}^{\infty}\left[ \frac{1}{16^i} \left( \frac{4}{8i + 1} - \frac{2}{8i + 4} - \frac{1}{8i + 5} - \frac{1}{8i + 6} \right) \right]. 
\end{align}

\subsection{Desafios:}
Alguns problemas fizeram com que o algorítimo, embora com conversão incrível, não tenha cumprido a sua função. Como a fórmula, diferente da descrição nos relatórios que encontramos retorna o n-ésimo termo de pi sem a necessidade de cálculo dos termos anteriores e não o n-ésimo algarismo.

Como este algorítimo gera uma somatória de números racionais, que por muitas vezes podem ser dízimas, fica muito dificil trabalhar com a precisão de um double em linguagem C muito menos usar o vetor de armazenamento de inteiros que foi proposto na descrição inicial do trabalho.

Embora o armazenamento da informação calculada seja perdido na segunda dezena de precisão é possível calcular os termos até que eles atinjam o valor máximo de precisão de uma variável double. No entanto isso ocorre no termo 107, e o algoritmo termina no termo 108.

Devido a isso não foi possível fazer a medição de tempo adequada para rodar em diversas maquinas (o método desenvolvido para a medição no primeiro ep é muito pouco preciso para este caso). Foi testada em um Acer Aspire com 8Gb de RAM e processador AMS Phenom II X4 Mobile (2GHz). E os resultados obtidos, calculando até o termo de maior precisão possível para uma variável long double foi de:

\begin{itemize}

\item \verb+SEQUENCIAL+: 5155175 ns
\item \verb+NORMAL+: 4238274 ns
\item \verb+UNLIMITED+: 1157542 ns

\end{itemize}

\subsection{Resultados:}

Foi possível observar que a velocidade de conversão tem uma boa melhora na versão concorrente, no entanto a versão sem sincronização a conversão ocorre de maneira muito mais veloz. A idéia original de salvar as casas decimais assim como o algoritmo promete não é viável, já que na segunda dezena de casas decimais a precisão já é perdida. E os valores dos termos, embora calculados corretamente, tenham suas dízimas arredondadas e faça com que o cálculo fique incorreto.

Acredito que os resultados estão de acordo com o esperado. E o desempenho do algoritmo é surpreendente. Devido a não ter completado os métodos de tomada de tempo em nanossegundos ficou impossível fazer os testes em mais máquinas (já que seriam insuficientes até então) e também fazer o gráfico de desempenho adequado.

\begin{thebibliography}{99}
\bibitem{wiki_pi}
\url{http://en.wikipedia.org/wiki/Pi}
\bibitem{wiki_bellard}
\url{http://en.wikipedia.org/wiki/Bellard%27s_formula}
\bibitem{wiki_bailey}
\url{http://en.wikipedia.org/wiki/Bailey-Borwein-Plouffe\_formula}
\end{thebibliography}

\end{document}
