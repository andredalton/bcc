\documentclass[12pt,a4paper]{article}
\usepackage[brazil]{babel}
\usepackage[utf8]{inputenc}
\usepackage[tmargin=2cm, bmargin=2cm, lmargin=2cm, rmargin=2cm]{geometry}
\usepackage{graphicx}
\usepackage{float}
\usepackage{indentfirst}
\usepackage[pdftex]{hyperref}
\usepackage{enumerate}
\usepackage{amsthm,amssymb,amstext,amsmath}
\usepackage{verbatim}
\usepackage{amsmath}
\usepackage{eucal}
\usepackage{mathrsfs}
\usepackage[utf8]{inputenc}
\usepackage{hyperref}
\usepackage{fancyvrb}
\DefineVerbatimEnvironment{code}{Verbatim}{fontsize=\small}
\DefineVerbatimEnvironment{example}{Verbatim}{fontsize=\small}

\title{Título\\MAC0438 - Programação Concorrente}
\author{
    André Meneghelli Vale - 4898948\\
    \texttt{andredalton@gmail.com}
    \and
    Marcello Souza de Oliveira - 6432692\\
    \texttt{mcellor210@gmail.com}
}
\date{}

\pdfinfo{%
  /Title    ()
  /Author   ()
  /Creator  ()
  /Producer ()
  /Subject  ()
  /Keywords ()
}

\begin{document}
\maketitle

\newpage

\section{Cálculo por série infinita de $\pi$}

\subsection{Fórmula:}
A fórmula escolhida foi a \emph{Fórmula de Bellard}\textsuperscript{[1]}. Esta escolha se deve ao fato de poder ser calculado um enésimo dígito de pi na base 2 sem a necessidade de se calcular qualquer um dos termos anteriores, o que torna o programa possível de ser totalmente paralelizável.
Esta fórmula foi desenvolvida a partir da fórmula \emph{Bailey–Borwein–Plouffe}\textsuperscript{[2]} (\verb+BBP+) que por sua vez foi inspirada na série infinita para o cálculo de uma aproximação da inversa da arcotangente.

\begin{align}
\pi = \frac1{2^6} \sum_{n=0}^\infty \frac{(-1)^n}{2^{10n}} \, \left(-\frac{2^5}{4n+1} \right. & {} - \frac1{4n+3} + \frac{2^8}{10n+1} - \frac{2^6}{10n+3} \left. {} - \frac{2^2}{10n+5} - \frac{2^2}{10n+7} + \frac1{10n+9} \right)
\end{align}

\begin{align}
\pi = \sum_{i = 0}^{\infty}\left[ \frac{1}{16^i} \left( \frac{4}{8i + 1} - \frac{2}{8i + 4} - \frac{1}{8i + 5} - \frac{1}{8i + 6} \right) \right]. 
\end{align}

\subsection{Vetor:}

Para o cálculo foi definida uma estrutura de \verb+N long int+ chamada de \verb+SuperLong+, cujo vetor principal estará no elemento \verb+data+:

\begin{code}
typedef struct superLong{
       unsigned long int *data;
       unsigned long int n;
} SuperLong;
\end{code}

\subsection{Thread:}

O objetivo é poder armazenar os dados calculados por cada uma das threads em uma estrutura de dados que não faz arredondamentos. As threads irão incrementar um buffer em variável \verb+long double+. Desta maneira não é necessário realizar uma conversão de base depois de terminado o cálculo, o que irá economizar uma quantidade considerável de tempo.

Cada vez que um limite de precisão que o vetor esta trabalhando é alcançado as threads ficam paradas em uma barreira enquanto a primeira thread faz a sincronização dos resultados e atualiza algumas variáveis calculadas durante a execução do programa. Essas variáveis serão explicadas na próxima sessão.

\subsection{Reaproveitando os termos da série:}

Podemos observar que o algorítimo escolhido repete muitas vezes algumas multiplicações e o cálculo da potência inicial. Desta maneira decidimos utilizar 5 variáveis globais que irão manter parte dos valores calculados a cada sincronização:

\begin{itemize}

\item \verb+N+: Número do ultimo termo a ser calculado no momento da sincronização;
\item \verb+N0+: Número do ultimo termo a ser calculado no momento da sincronização anterior. Inicializada com \emph{0};
\item \verb+m4+: Inicializada com \emph{0} e incrementada a cada sincronização com \emph{(N-N0)*4}. Desta forma os denominadores terão o formato \emph{m4+4*(n-N)+k}, onde \verb+n+ é o termo que a thread esta calculando e \verb+k+ é a constante presente em cada um dos denominadores;
\item \verb+m10+: Inicializada com \emph{0} e incrementada a cada sincronização com \emph{(N-N0)*10}. Desta forma os denominadores terão o formato \emph{m10+10*(n-N)+k}, seguindo a mesma representação do ítem anterior;
\item \verb+p2+: Inicializada com \emph{$2^6$} e multuplicada a cada sincronização por \emph{$2^{(N-N0)*10}$}. Então no final do cálculo o termo, até agora contendo a soma das parcelas, será dividido por \emph{$(p2*2^{10(n-N)})$};

\end{itemize}

Outro cálculo que é desnecessário é o de $(-1)^n$, esta parte será substituida por um \verb+if+ para definir a parcela negativa em caso de \verb+n+ ímpar e positiva caso contrário.

\begin{thebibliography}{99}
\bibitem{wiki_pi}
\url{http://en.wikipedia.org/wiki/Pi}
\bibitem{wiki_bellard}
\url{http://en.wikipedia.org/wiki/Bellard%27s_formula}
\bibitem{wiki_bailey}
\url{http://en.wikipedia.org/wiki/Bailey-Borwein-Plouffe\_formula}
\end{thebibliography}

\end{document}
