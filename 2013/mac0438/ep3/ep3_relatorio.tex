\documentclass[12pt,a4paper]{article}
\usepackage[brazil]{babel}
\usepackage[utf8]{inputenc}
\usepackage[tmargin=2cm, bmargin=2cm, lmargin=2cm, rmargin=2cm]{geometry}
\usepackage{graphicx}
\usepackage{float}
\usepackage{indentfirst}
\usepackage[pdftex]{hyperref}
\usepackage{enumerate}
\usepackage{amsthm,amssymb,amstext,amsmath}
\usepackage{verbatim}
\usepackage{amsmath}
\usepackage{eucal}
\usepackage{mathrsfs}
\usepackage[utf8]{inputenc}
\usepackage{hyperref}
\usepackage{fancyvrb}
\DefineVerbatimEnvironment{code}{Verbatim}{fontsize=\small}
\DefineVerbatimEnvironment{example}{Verbatim}{fontsize=\small}

\title{Título\\MAC0438 - Programação Concorrente}
\author{
    André Meneghelli Vale - 4898948\\
    \texttt{andredalton@gmail.com}
    \and
    Marcello Souza de Oliveira - 6432692\\
    \texttt{mcellor210@gmail.com}
}
\date{}

\pdfinfo{%
  /Title    ()
  /Author   ()
  /Creator  ()
  /Producer ()
  /Subject  ()
  /Keywords ()
}

\begin{document}
\maketitle

\newpage

\section{Problema das abelhas e ursos.}

\subsection{Descrição do problema:}

O problema deste EP deve ser resolvido utilizando monitores para controlar a concorrência. Os monitores que devem ser implementados serão o \verb+abelha+ e o \verb+urso+. Suas funções principais de controle de fluxo estão descritas nas duas próximas subsessões e a ultima subsessão faz uma abordagem ao funcionamento dos monitores em conjunto.

\subsection{Abelhas:}

Como descrito no problema existem \verb+N+ \emph{abelhas caridosas} que alimentam \verb+B+ \emph{ursos enjaulados}. Para resolver o problema o monitor abelha será implementado com uma variável \verb+tatu+, em memória compartilhada, funcionando como um contador regressivo. Esta variável indica quanto tempo resta para que cada uma das abelhas produza mel. Este contador será decrementado de maneira sincronizada entre os monitores do tipo \emph{abelha} utilizando a função \verb+sincronizacao+ implementada por ele.

Imediatamente antes do decréscimo do tempo restante para que esta abelha produza mel haverá a chamada á função \verb+sincronizacao+ que espera que todas as abelhas estejam sincronizadas neste ponto. Para tanto o monitor abelha terá uma variável \verb+Nab+ em memória compartilhada que guarda o número restante de processos que faltam a ser sincronizados. Imediatamente antes de chamar \verb+sincronizacao+ o monitor irá decrementar o número de processos ainda não sincronizados e \verb+sincronizacao+ irá aguardar que o contador \verb+Nab+ atinga o valor \verb+0+.

Como ainda há necessidade de as abelhas estarem paradas enquanto um urso se alimenta existe também uma variável \verb+sleep+, em memória compartilhada, que irá indicar se as abelhas podem ou não prosseguir. Seu valor será alterado através das funções \verb+allSleep+ (\verb+sleep=TRUE+) e \verb+allWake+ (\verb+sleep=FALSE+) implementadas no monitor abelha.

Então para contemplar as duas necessidades a função \verb+sincronizacao+ irá aguardar que todos os processos estejam sincronizados e as abelhas não estejam dormindo para então prosseguir com o algoritmo.

Uma vez que o contador regressivo chegue a \verb+0+ a abelha irá incrementar o número de porções de mel contidas no pote e irá verificar se o pote esta cheio. Se o pote estiver cheio esta abelha chama a função \verb+allSleep+ e logo após \verb+signal+ do monitor urso para que um dos ursos seja acordado.

\subsection{Ursos:}

O monitor urso controla as \verb+B+ threads ursos. 
Uma variável \verb+b em [0:B-1]+ controla o urso que será acordado (mantendo a justiça no acesso dos ursos ao pote). 
Isto é, no monitor urso, um wait aguarda que o b-ésimo urso seja acordado, quando esse urso acorda, no primeiro momento, decrementa a variável \verb+H+ à metade, e depois para \verb+zero+ (pulando no 'tempo virtual' para os momentos relevantes). Como apenas um urso pode comer por vez, só temos uma thread ativa nesse momento, e portanto, não precisamos sincronizar o tempo.

Neste momento, chama a \verb+allWake+ para acordar as abelhas e volta a dormir (note que a variável b é incrementada para que na próxima chamada \verb+allSleep+ das abelhas, um urso diferente seja acordado).

\subsection{Monitores trabalhando em conjunto:}

\end{document}
