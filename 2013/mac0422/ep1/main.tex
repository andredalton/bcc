\documentclass[12pt,a4paper]{article}
\usepackage[brazil]{babel}
\usepackage[utf8]{inputenc}
\usepackage[tmargin=2cm, bmargin=2cm, lmargin=2cm, rmargin=2cm]{geometry}
\usepackage{graphicx}
\usepackage{float}
\usepackage{indentfirst}
\usepackage[pdftex]{hyperref}
\usepackage{enumerate}
\usepackage{bbm,amsthm,amssymb,amstext,amsmath}
\usepackage{verbatim}
\usepackage{listings}

\author{
    André Meneghelli Vale - 4898948\\
    \texttt{andredalton@gmail.com}
    \and
    Hilder Vitor Lima Pereira - 6777064\\
    \texttt{vitor\_lp@yahoo.com}
}

\date{01 de Setembro} % uma data antiga para fingir que estava sendo feito há muito tempo hahahaha

\title{Relatório: EP1}

\begin{document}
\maketitle

\section{Introdução}

\subsection{Minix}

   O \emph{Minix} é um sistema operacional semelhante ao \emph{Unix} gratuito e com o código fonte disponível em suas distribuições. É escrito em \emph{Linguagem C} e \emph{assembly}. \emph{Andrew S. Tanenbaum} criou este sistema para explicar os principios de funcionamento de seu livro \emph{"Operating Systems Design and Implementation"}.

   As vantagens deste sistema, além de ser disponibilizado com o seu código fonte, são a necessidade muito reduzida de memória RAM e disco rígido quando comparado aos sistemas operacionais utilizados atualmente e uma arquitetura interessante para o aprendizado. Uma vez que os processos são entidades independentes e estão restritos a camadas, cada processo tem as suas permissões de acesso e algumas propriedades.

   É possível encontrar mais informações e baixar as várias versões disponíveis deste sistema em \url{http://www.minix3.org/}. Para este trabalho foi escolhida a versão \href{http://www.minix3.org/iso/minix_R3.1.7-r7256.iso.gz}{3.1.7}.

\subsection{Virtualbox}

   Para facilitar a instalação e distribuição das alterações necessárias para este Exercício foi estipulado o uso do software de virtualização \href{https://www.virtualbox.org/}{Virtualbox}.

   Este software é gratuito e compatível com vários sistemas operacionais atuais. Outra característica importante é a capacidade de criação de pastas compartilhadas entre a maquina virtual e o hospedeiro, o que facilita muito a criação de um bom ambiente de programação.
   
   Para realizar a configuração do sistema operacional de maneira mais conveniente optamos por configurar a placa de rede da VM em modo NAT e usamos redirecionamento de portas para fazer conexão ssh. Permitindo desta maneira que todo o Minix fosse editado usando as ferramentas preferidas de cada um dos integrantes do grupo.

\subsection{Problema proposto}

   Modificar o sistema, fazendo com que um resumo da tabela de processos seja mostrada quando a tecla \verb+F5+ for acionada. Este resumo deve conter as informações na ordem da lista a seguir:

\begin{itemize}
\item PID: identificador do processo;
\item Tempo de cpu;
\item Tempo de sistema;
\item Tempo dos filhos;
\item Endereço do ponteiro da pilha e dos segmentos data, bss e text;
\end{itemize}

\section{Códigos alterados}

\subsection{dmp.c}

   Este arquivo contém o mapeamento de caracteres, foi usado para poder tratar a captura de interrupção da tecla \emph{F5}.

\subsubsection{Localização}

   Diretório: /usr/src/servers/is/

\subsubsection{Alterações [23-27]}
   
   \lstinputlisting[language=C, firstline=23, lastline=27]{dmp.c}

\subsection{proto.h}

   Arquivo com os protótipos das funções usadas no arquivo \emph{dmp\_kernel.c}.

\subsubsection{Localização}

   Diretório: /usr/src/servers/is/

\subsubsection{Alterações [14-18]}
   
   \lstinputlisting[language=C, firstline=14, lastline=18]{proto.h}

\subsection{/usr/src/servers/is/dmp\_kernel.c}

   Este arquivo contém a função alterada \emph{ custom\_proctab\_dmp()} que faz a impressão dos processos.

\subsubsection{Localização}

   Diretório /usr/src/servers/is/
   
\subsubsection{Alterações [14-18] [415-585]}

A alteração a seguir serve apenas para ter acesso a estrura mproc dentro do escopo do arquivo atual.
\lstinputlisting[language=C, firstline=14, lastline=18]{dmp_kernel.c}

\section{Conclusão}

\end{document}