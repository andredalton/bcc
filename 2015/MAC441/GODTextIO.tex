\section{Textos (GODTextIO)}

O \verb|GODTextIO| é o módulo integrante do GOD (Grande Organizador de
Dados) que se encarrega da conversão do conteúdo arquivos texto para
um objeto do tipo \verb|GODData| e vice-versa. Atualmente, o módulo
lida com arquivos \verb|odt|, \verb|pdf|, \verb|rtf|, \verb|doc|, \verb|docx| e \verb|txt|. A seguir, daremos uma descrição sobre os pré-requisitos para seu
funcionamento e, logo depois, a composição do módulo, além de mostrar
como lidamos com os diversos tipos de arquivos.

\subsection{Equipe}

Equipe original: João Paulo Camargo, John
Gardenghi\footnote{\href{mailto:john@ime.usp.br}{john@ime.usp.br}},
Marcello Oliveira, José Eurípedes.
Equipe 2015: André M. Vale, Evandro F. Giovanini


\subsection{Pré-requisitos}

Para que este módulo funcione corretamente, é necessário
\begin{itemize}
\item Ter o pacote \verb|OSProcess| instalado no Squeak. O \verb|OSProcess| é usado para executar comandos no terminal do SO. Ele está incluído nos pacotes dependências do repositório do GOD.
\item Ter o \verb|pdflatex| e o \verb|unoconv| devidamente instalados e funcionando por chamada no terminal.
\end{itemize}

\subsection{Classes}

\vspace{1em}
\textbf{TXTIOConverter}

É a interface com o GOD. Qualquer outro módulo que deseje usar alguma
das nossas funcionalidades, deve fazer chamadas a métodos desta
classe. Ela é apenas a porta de entrada do nosso módulo, portanto, as
funcionalidades que ela implementa não estão codificadas nesta classe,
mas apenas são feitas chamadas a métodos competentes para cada
funcionalidade.

Esta classe possui apenas um atributo, que é o \verb|types|. É um
dicionário cujas chaves são os tipos suportados pelo módulo, a saber
\verb|odt|, \verb|pdf|, \verb|rtf| e \verb|txt|, e os valores são as
classes associadas a cada tipo de arquivo. Por outro lado, temos dez
métodos. A seguir, descrevemos os que interessam a outros módulos do
GOD. Logo depois, mostramos os demais métodos da classe, usados para
finalidades específicas da própria classe.
\begin{enumerate}

\item \verb|readFromFile: aPath type: fileType| é responsável por ler
  um arquivo cujo caminho é passado no argumento \verb|aPath| e cujo
  tipo é passado no argumento \verb|fileType|. Os tipos de arquivos
  válidos são \verb|odt|, \verb|pdf|, \verb|rtf| e \verb|txt|. Devolve
  um objeto do tipo \verb|GODData|. Sua variação é o método
  \verb|readFromFile: aPath|, em que o tipo do arquivo é extraído do
  caminho passado.

\item
  \verb|write: godData toFile: aPath type: fileType append: app overwrite: ovwr|
  é responsável por escrever o conteúdo do objeto \verb|godData| do
  tipo \verb|GODData| em um arquivo cujo caminho é passado pelo
  argumento \verb|aPath| do tipo \verb|fileType|. Os argumentos (a)
  \verb|append| e (b) \verb|overwrite| são valores lógicos que
  determinam, caso o arquivo a ser escrito já exista, se (a) o novo
  conteúdo deve ser adicionado ao final do arquivo já existente e (b)
  o arquivo deve ser sobrescrito. Suas variações são quatro métodos,
  variando apenas a combinação dos argumentos, que são
  \begin{itemize}

  \item \verb|write: godData toFile: aPath|,
  \item \verb|write: godData toFile: aPath type: fileType|,
  \item \verb|write: godData toFile: aPath type: fileType append: app|,
  \item \verb|write: godData toFile: aPath type: fileType overwrite: ovwr|.

  \end{itemize}

\end{enumerate}

Além desses sete métodos, temos
\begin{enumerate}

\item \verb|getFileTypeFromPath: aPath| extrai o tipo do arquivo a
  partir do caminho passado como argumento em \verb|aPath| e o
  retorna.

\item \verb|getPossibleTypes| retorna uma lista com as chaves do
  dicionário \verb|types|.

\item \verb|initialize| inicializa o dicionário \verb|types|.

\end{enumerate}

\vspace{1em}
\textbf{TXTMainConverter}

É a classe mãe que determina os métodos que cada classe filha deve
implementar de forma ler ou escrever arquivos de tipos
específicos. Possui sete atributos:
\begin{enumerate}
\item \verb|append| determina se o conteúdo novo deve ser adicionado
  ao final de um arquivo existente (lógico),
\item \verb|author| contém o autor do conteúdo (texto),
\item \verb|contents| contém o conteúdo do arquivo (texto),
\item \verb|date| contém a data (texto),
\item \verb|overwrite| determina se o arquivo deve ser sobrescrito
  (lógico),
\item \verb|path| contém o caminho do arquivo (texto),
\item \verb|title| contém o título (texto).
\end{enumerate}

Contém dois métodos principais:
\begin{enumerate}
\item \verb|readFile| lê um arquivo. Nesta classe, possui uma
  implementação genérica, em que converte o tipo do arquivo para
  \verb|odt|, usando o \verb|unoconv|, e lê este arquivo usando o
  leitor implementado na classe \verb|TXTIOOdtConverter|. Todavia, é
  sobrescrito pelas classes filhas \verb|TXTIOOdtConverter| e
  \verb|TXTIOTextConverter|. O conteúdo lido do arquivo é salvo no
  atributo \verb|content| na classe, e este método retorna
  \verb|true|.

\item \verb|writeFile| é apenas um método abstrato.
\end{enumerate}

Além disso, temos \verb|convertFileType|, que faz a conversão do tipo
do arquivo usando o \verb|unoconv|, e os métodos de acesso aos
atributos.

As classes que lidam com os tipos suportados são:
\begin{enumerate}
\item \verb|TXTIOOdtConverter| lida com arquivos \verb|odt|. Explora o
  padrão \textit{OpenDocument Format} (\verb|odf|). Esse formato é uma
  forma compactada de alguns arquivos, dos quais nos interessam o
  arquivo \verb|content.xml|. Na classe \verb|TXTOdtConverter|, o
  método \verb|readFile|, se encarrega primeiro de chamar o
  \verb|extractTextXML| para extrair o conteúdo interessante
  (\textit{tags} office:*) desse arquivo, primeiro descompactando o
  formato \verb|odt| (através da classe \verb|ZipArchive| do Squeak) e
  depois lendo como uma \textit{stream}. Depois o \verb|readFile|
  extrai a informação útil dessa \textit{stream}, verificando as
  \textit{tags}. O \verb|writeFile|, cria um diretório temporário e os
  arquivos que compoem o \verb|odt|, e escreve no \verb|content.xml| o
  conteúdo (conforme o formato \verb|xml|), compacta para \verb|odt| e
  o move para o caminho desejado pelo usuário.
\item \verb|TXTIOPdfConventer| lida com arquivos \verb|pdf|. Escreve
  arquivos usando o \verb|pdflatex| e lê usando o conversor do
  \verb|unoconv| implementado na classe mãe.
\item \verb|TXTIORtfConverter| lida com arquivos \verb|rtf|. Para a
  escrita, criamos algumas tags básicas do formato e escrevemos,
  através da TXTTextConverter, em um arquivo, no formato rtf, todavia
  os lê usando o conversor \verb|unoconv| implementado na classe mãe.
\item \verb|TXTIODocConverter| lida com arquivos \verb|doc|. A escrita também é feita de forma básica, e a leitura usa o suporte implementado no \verb|unoconv|.
\item \verb|TXTIODocxConverter| lida com arquivos \verb|docx|. Funciona de forma análoga a \verb|TXTIODocConverter|.
\item \verb|TXTIOTextConverter| lida com arquivos \verb|txt|. Escreve
  e lê os arquivos usando a classe \verb|FileStream| do Squeak
  Smalltalk.
\end{enumerate}

\subsection{Exemplos}

\begin{godCode}
'Criando um arquivo texto a partir de um GODData' 
txtio:=TXTIOConverter new.
txtio write: godData toFile:'/path/to/file.txt'.

'Lendo um arqwuivo ODT e adicionando a um GODData'
gOdt:=txtio readFromFile:'/path/to/file.odt'.

\end{godCode}
